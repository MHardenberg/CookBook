\documentclass[10pt]{book}
\usepackage[margin=18mm]{geometry}
\usepackage{setup}

\title{\sc the Struggeling Chef's Cook Book}
\author{Michel}
\date{\today}

\begin{document}
\maketitle

\tableofcontents

\section{To Do}{
  \begin{itemize}
    \item Genral tips and debatable trueisms
      \begin{itemize}
        \item Boiling over frying
        \item Unpack carrot and potatos so that they dry instead of spoiling
        \item Prefer basic seasoning - simple lets the vegetables
          define the taste
        \item Add salt early - herbs and pepper late or after adding
          water to preserve flavour
        \item never fry garlic if you want garlic taste.
        \item Salads are a waste of money - they don't fill you, go bad easy and
          your body is bad at breaking down uncooked vegtables anyways.
        \item Don't buy milk unless you specifically need it. Oats
          are fine without. We're living in a recession.
        \item slight variations can make you think you eat different meals.
        \item Prefer course oats and wholegrain flour as they are more filling.
        \item Olive oil is for the upper class.
        \item Always boil extra rice, potato or pasta for later use.
      \end{itemize}

    \item Dumpster diving
    \item Poor man's mushroom pan
    \item Saurkraut
    \item the everthing soup (incl. variations)
    \item Makebelieve Mashed Potato
    \item European Instant noodles (Reasonably fast noodles)
    \item Potato Salad
    \item Mayonaise
    \item Jam Juice
    \item Leftover's vegtable stock
    \item Backed Onion
    \item Euro Fried Rice
    \item Simple potato pan
    \item Holiday oats
    \item The fridge raid (fried everything with potato or pasta)
    \item Poor knights
    \item Pancakes the correct way
    \item Buy barley, buck-wheat and the like in bulk from bakery suppliers.
  \end{itemize}

}

\end{document}
